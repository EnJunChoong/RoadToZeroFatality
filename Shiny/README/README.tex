\documentclass[]{article}
\usepackage{lmodern}
\usepackage{amssymb,amsmath}
\usepackage{ifxetex,ifluatex}
\usepackage{fixltx2e} % provides \textsubscript
\ifnum 0\ifxetex 1\fi\ifluatex 1\fi=0 % if pdftex
  \usepackage[T1]{fontenc}
  \usepackage[utf8]{inputenc}
\else % if luatex or xelatex
  \ifxetex
    \usepackage{mathspec}
  \else
    \usepackage{fontspec}
  \fi
  \defaultfontfeatures{Ligatures=TeX,Scale=MatchLowercase}
\fi
% use upquote if available, for straight quotes in verbatim environments
\IfFileExists{upquote.sty}{\usepackage{upquote}}{}
% use microtype if available
\IfFileExists{microtype.sty}{%
\usepackage{microtype}
\UseMicrotypeSet[protrusion]{basicmath} % disable protrusion for tt fonts
}{}
\usepackage[margin=1in]{geometry}
\usepackage{hyperref}
\hypersetup{unicode=true,
            pdftitle={README},
            pdfborder={0 0 0},
            breaklinks=true}
\urlstyle{same}  % don't use monospace font for urls
\usepackage{color}
\usepackage{fancyvrb}
\newcommand{\VerbBar}{|}
\newcommand{\VERB}{\Verb[commandchars=\\\{\}]}
\DefineVerbatimEnvironment{Highlighting}{Verbatim}{commandchars=\\\{\}}
% Add ',fontsize=\small' for more characters per line
\usepackage{framed}
\definecolor{shadecolor}{RGB}{248,248,248}
\newenvironment{Shaded}{\begin{snugshade}}{\end{snugshade}}
\newcommand{\KeywordTok}[1]{\textcolor[rgb]{0.13,0.29,0.53}{\textbf{#1}}}
\newcommand{\DataTypeTok}[1]{\textcolor[rgb]{0.13,0.29,0.53}{#1}}
\newcommand{\DecValTok}[1]{\textcolor[rgb]{0.00,0.00,0.81}{#1}}
\newcommand{\BaseNTok}[1]{\textcolor[rgb]{0.00,0.00,0.81}{#1}}
\newcommand{\FloatTok}[1]{\textcolor[rgb]{0.00,0.00,0.81}{#1}}
\newcommand{\ConstantTok}[1]{\textcolor[rgb]{0.00,0.00,0.00}{#1}}
\newcommand{\CharTok}[1]{\textcolor[rgb]{0.31,0.60,0.02}{#1}}
\newcommand{\SpecialCharTok}[1]{\textcolor[rgb]{0.00,0.00,0.00}{#1}}
\newcommand{\StringTok}[1]{\textcolor[rgb]{0.31,0.60,0.02}{#1}}
\newcommand{\VerbatimStringTok}[1]{\textcolor[rgb]{0.31,0.60,0.02}{#1}}
\newcommand{\SpecialStringTok}[1]{\textcolor[rgb]{0.31,0.60,0.02}{#1}}
\newcommand{\ImportTok}[1]{#1}
\newcommand{\CommentTok}[1]{\textcolor[rgb]{0.56,0.35,0.01}{\textit{#1}}}
\newcommand{\DocumentationTok}[1]{\textcolor[rgb]{0.56,0.35,0.01}{\textbf{\textit{#1}}}}
\newcommand{\AnnotationTok}[1]{\textcolor[rgb]{0.56,0.35,0.01}{\textbf{\textit{#1}}}}
\newcommand{\CommentVarTok}[1]{\textcolor[rgb]{0.56,0.35,0.01}{\textbf{\textit{#1}}}}
\newcommand{\OtherTok}[1]{\textcolor[rgb]{0.56,0.35,0.01}{#1}}
\newcommand{\FunctionTok}[1]{\textcolor[rgb]{0.00,0.00,0.00}{#1}}
\newcommand{\VariableTok}[1]{\textcolor[rgb]{0.00,0.00,0.00}{#1}}
\newcommand{\ControlFlowTok}[1]{\textcolor[rgb]{0.13,0.29,0.53}{\textbf{#1}}}
\newcommand{\OperatorTok}[1]{\textcolor[rgb]{0.81,0.36,0.00}{\textbf{#1}}}
\newcommand{\BuiltInTok}[1]{#1}
\newcommand{\ExtensionTok}[1]{#1}
\newcommand{\PreprocessorTok}[1]{\textcolor[rgb]{0.56,0.35,0.01}{\textit{#1}}}
\newcommand{\AttributeTok}[1]{\textcolor[rgb]{0.77,0.63,0.00}{#1}}
\newcommand{\RegionMarkerTok}[1]{#1}
\newcommand{\InformationTok}[1]{\textcolor[rgb]{0.56,0.35,0.01}{\textbf{\textit{#1}}}}
\newcommand{\WarningTok}[1]{\textcolor[rgb]{0.56,0.35,0.01}{\textbf{\textit{#1}}}}
\newcommand{\AlertTok}[1]{\textcolor[rgb]{0.94,0.16,0.16}{#1}}
\newcommand{\ErrorTok}[1]{\textcolor[rgb]{0.64,0.00,0.00}{\textbf{#1}}}
\newcommand{\NormalTok}[1]{#1}
\usepackage{graphicx,grffile}
\makeatletter
\def\maxwidth{\ifdim\Gin@nat@width>\linewidth\linewidth\else\Gin@nat@width\fi}
\def\maxheight{\ifdim\Gin@nat@height>\textheight\textheight\else\Gin@nat@height\fi}
\makeatother
% Scale images if necessary, so that they will not overflow the page
% margins by default, and it is still possible to overwrite the defaults
% using explicit options in \includegraphics[width, height, ...]{}
\setkeys{Gin}{width=\maxwidth,height=\maxheight,keepaspectratio}
\IfFileExists{parskip.sty}{%
\usepackage{parskip}
}{% else
\setlength{\parindent}{0pt}
\setlength{\parskip}{6pt plus 2pt minus 1pt}
}
\setlength{\emergencystretch}{3em}  % prevent overfull lines
\providecommand{\tightlist}{%
  \setlength{\itemsep}{0pt}\setlength{\parskip}{0pt}}
\setcounter{secnumdepth}{0}
% Redefines (sub)paragraphs to behave more like sections
\ifx\paragraph\undefined\else
\let\oldparagraph\paragraph
\renewcommand{\paragraph}[1]{\oldparagraph{#1}\mbox{}}
\fi
\ifx\subparagraph\undefined\else
\let\oldsubparagraph\subparagraph
\renewcommand{\subparagraph}[1]{\oldsubparagraph{#1}\mbox{}}
\fi

%%% Use protect on footnotes to avoid problems with footnotes in titles
\let\rmarkdownfootnote\footnote%
\def\footnote{\protect\rmarkdownfootnote}

%%% Change title format to be more compact
\usepackage{titling}

% Create subtitle command for use in maketitle
\providecommand{\subtitle}[1]{
  \posttitle{
    \begin{center}\large#1\end{center}
    }
}

\setlength{\droptitle}{-2em}

  \title{README}
    \pretitle{\vspace{\droptitle}\centering\huge}
  \posttitle{\par}
    \author{}
    \preauthor{}\postauthor{}
      \predate{\centering\large\emph}
  \postdate{\par}
    \date{May 24, 2019}


\begin{document}
\maketitle

\subsubsection{About}\label{about}

Welcome! Here on this page, you can find the details about the bit and
pieces that were used to create this app. We will also provide a brief
guidelines on the features of this app. This app was developed as a
assignment for University of Malaya's Principles of Data Science
(WQD7001) Master coursework. Our ultimate goal is however to provide a
platform to easily visualize UK fatal road accidents data. We hope that
this is beneficial to the users and can serve as an example for future
work.

\subsubsection{Features on Map
Visualization}\label{features-on-map-visualization}

Under Map Visualization tab, We can filter the data set to show only
relevant years or months on the map. The visualization on the map are
clustered. When not clustered, a single red circle marker represent a
fatal accident. There is a popup information table when clicked on the
red circle marker.

If we wish to zoom into the map based on district level, there is a
separate select input to do so:

And if we wish to view specific details on a given Accident index, we
can input the accident index ID on the given textbox. The result will be
a zoomed in view to the accident location, and a table at the bottom of
the map.

\subsubsection{Features on Chart
Visualization}\label{features-on-chart-visualization}

On the next tab is the tab for Chart Visualization. On this tab, we
designed it in a way for users to select chart types according to their
preference. Three Chart options namely: PieChart, Histogram and
2DHistogram are allowed. Users can select the Attributes desired to be
plotted on the chart.

For 2D Histogram that needs a second attribute, an additional attribute
selection is triggered when user select 2D Histogram as the desired
Chart Type. A Color Scale option is also available for user to select
the desired color on the 2DHistogram.

\subsubsection{Datasets}\label{datasets}

\begin{itemize}
\item
  UK 2005-2015 road accidents datasets datasets
  \url{https://www.kaggle.com/silicon99/dft-accident-data}
  \textless{}Accidents0515.csv\textgreater{},\textless{}Vehicles0515.csv\textgreater{}
\item
  Road-Accidet-Safety-Data-Guide
  \url{http://data.dft.gov.uk/road-accidents-safety-data/Road-Accident-Safety-Data-Guide.xls}
  \textless{}Road-Accident-Safety-Data-Guide.xls\textgreater{}
\item
  UK Map Polygons geojson file
  \url{https://blog.exploratory.io/making-maps-for-uk-countries-and-local-authorities-areas-in-r-b7d222939597}
  \textless{}uk\_la.geojson\textgreater{}
\end{itemize}

\subsubsection{Datasets Cleaning And
Processing}\label{datasets-cleaning-and-processing}

A simple EDA was done in the beginning using

\begin{Shaded}
\begin{Highlighting}[]
\KeywordTok{head}\NormalTok{(), }\KeywordTok{summary}\NormalTok{(), }\KeywordTok{str}\NormalTok{(), }\KeywordTok{boxplot}\NormalTok{()}
\end{Highlighting}
\end{Shaded}

After identifying NA and missing variables, we remove rows with all NA
cells, and imputed individual NAs in attribute columns as `Data Missing'

\begin{Shaded}
\begin{Highlighting}[]
\CommentTok{#Rows with All NAs have less than 8 characters in Accident_Index }
\NormalTok{myAccidents=myAccidents[}\OperatorTok{!}\NormalTok{(}\KeywordTok{nchar}\NormalTok{(}\KeywordTok{as.character}\NormalTok{(myAccidents}\OperatorTok{$}\NormalTok{Accident_Index))}\OperatorTok{<}\DecValTok{8}\NormalTok{),]}
\NormalTok{myVehicles=myVehicles[}\OperatorTok{!}\NormalTok{(}\KeywordTok{nchar}\NormalTok{(}\KeywordTok{as.character}\NormalTok{(myVehicles}\OperatorTok{$}\NormalTok{Accident_Index))}\OperatorTok{<}\DecValTok{8}\NormalTok{),]}

\NormalTok{myAccidents[}\KeywordTok{is.na}\NormalTok{(myAccidents)]=}\StringTok{ }\OperatorTok{-}\DecValTok{1} 
\NormalTok{myVehicles[}\KeywordTok{is.na}\NormalTok{(myVehicles)]=}\StringTok{ }\OperatorTok{-}\DecValTok{1}
\end{Highlighting}
\end{Shaded}

Since we are only interested with accidents that have fatalities, we
filter the data based on Accident Severity

\begin{Shaded}
\begin{Highlighting}[]
\NormalTok{cleanAccidents =}\StringTok{ }\NormalTok{myAccidents}\OperatorTok\KeywordTok{filter}\NormalTok{(myAccidents}\OperatorTok{$}\NormalTok{AccidentSeverity }\OperatorTok{==}\DecValTok{1}\NormalTok{)}
\NormalTok{cleanVehicles =}\StringTok{ }\NormalTok{myVehicles}\OperatorTok{%>%%>%}\KeywordTok{filter}\NormalTok{(myAccidents}\OperatorTok{$}\NormalTok{AccidentSeverity }\OperatorTok{==}\DecValTok{1}\NormalTok{)}
\end{Highlighting}
\end{Shaded}

Next we join both datasets on Vehicles rows according to accident index

\begin{Shaded}
\begin{Highlighting}[]
\NormalTok{cleanDataset =}\StringTok{ }\KeywordTok{left_join}\NormalTok{(myVehicles,myAccidents)}
\end{Highlighting}
\end{Shaded}

We dropped attributes that we think were irrelevant to visualization,
and plug that into the shiny app. The shiny source code can be found on
github.

\subsubsection{R packages used}\label{r-packages-used}

\begin{itemize}
\tightlist
\item
  Shiny \url{https://github.com/rstudio/shiny}
\item
  Leaflet \url{https://github.com/rstudio/leaflet}
\item
  tidyverse \url{https://github.com/tidyverse/tidyverse}
\item
  shinythemes \url{https://github.com/rstudio/shinythemes}
\item
  kableExtra \url{https://github.com/haozhu233/kableExtra}
\item
  plotly \url{https://github.com/ropensci/plotly}
\item
  ggthemes \url{https://github.com/jrnold/ggthemes}
\end{itemize}


\end{document}
